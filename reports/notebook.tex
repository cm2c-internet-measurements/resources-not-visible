
% Default to the notebook output style

    


% Inherit from the specified cell style.




    
\documentclass[11pt]{article}

    
    
    \usepackage[T1]{fontenc}
    % Nicer default font (+ math font) than Computer Modern for most use cases
    \usepackage{mathpazo}

    % Basic figure setup, for now with no caption control since it's done
    % automatically by Pandoc (which extracts ![](path) syntax from Markdown).
    \usepackage{graphicx}
    % We will generate all images so they have a width \maxwidth. This means
    % that they will get their normal width if they fit onto the page, but
    % are scaled down if they would overflow the margins.
    \makeatletter
    \def\maxwidth{\ifdim\Gin@nat@width>\linewidth\linewidth
    \else\Gin@nat@width\fi}
    \makeatother
    \let\Oldincludegraphics\includegraphics
    % Set max figure width to be 80% of text width, for now hardcoded.
    \renewcommand{\includegraphics}[1]{\Oldincludegraphics[width=.8\maxwidth]{#1}}
    % Ensure that by default, figures have no caption (until we provide a
    % proper Figure object with a Caption API and a way to capture that
    % in the conversion process - todo).
    \usepackage{caption}
    \DeclareCaptionLabelFormat{nolabel}{}
    \captionsetup{labelformat=nolabel}

    \usepackage{adjustbox} % Used to constrain images to a maximum size 
    \usepackage{xcolor} % Allow colors to be defined
    \usepackage{enumerate} % Needed for markdown enumerations to work
    \usepackage{geometry} % Used to adjust the document margins
    \usepackage{amsmath} % Equations
    \usepackage{amssymb} % Equations
    \usepackage{textcomp} % defines textquotesingle
    % Hack from http://tex.stackexchange.com/a/47451/13684:
    \AtBeginDocument{%
        \def\PYZsq{\textquotesingle}% Upright quotes in Pygmentized code
    }
    \usepackage{upquote} % Upright quotes for verbatim code
    \usepackage{eurosym} % defines \euro
    \usepackage[mathletters]{ucs} % Extended unicode (utf-8) support
    \usepackage[utf8x]{inputenc} % Allow utf-8 characters in the tex document
    \usepackage{fancyvrb} % verbatim replacement that allows latex
    \usepackage{grffile} % extends the file name processing of package graphics 
                         % to support a larger range 
    % The hyperref package gives us a pdf with properly built
    % internal navigation ('pdf bookmarks' for the table of contents,
    % internal cross-reference links, web links for URLs, etc.)
    \usepackage{hyperref}
    \usepackage{longtable} % longtable support required by pandoc >1.10
    \usepackage{booktabs}  % table support for pandoc > 1.12.2
    \usepackage[inline]{enumitem} % IRkernel/repr support (it uses the enumerate* environment)
    \usepackage[normalem]{ulem} % ulem is needed to support strikethroughs (\sout)
                                % normalem makes italics be italics, not underlines
    

    
    
    % Colors for the hyperref package
    \definecolor{urlcolor}{rgb}{0,.145,.698}
    \definecolor{linkcolor}{rgb}{.71,0.21,0.01}
    \definecolor{citecolor}{rgb}{.12,.54,.11}

    % ANSI colors
    \definecolor{ansi-black}{HTML}{3E424D}
    \definecolor{ansi-black-intense}{HTML}{282C36}
    \definecolor{ansi-red}{HTML}{E75C58}
    \definecolor{ansi-red-intense}{HTML}{B22B31}
    \definecolor{ansi-green}{HTML}{00A250}
    \definecolor{ansi-green-intense}{HTML}{007427}
    \definecolor{ansi-yellow}{HTML}{DDB62B}
    \definecolor{ansi-yellow-intense}{HTML}{B27D12}
    \definecolor{ansi-blue}{HTML}{208FFB}
    \definecolor{ansi-blue-intense}{HTML}{0065CA}
    \definecolor{ansi-magenta}{HTML}{D160C4}
    \definecolor{ansi-magenta-intense}{HTML}{A03196}
    \definecolor{ansi-cyan}{HTML}{60C6C8}
    \definecolor{ansi-cyan-intense}{HTML}{258F8F}
    \definecolor{ansi-white}{HTML}{C5C1B4}
    \definecolor{ansi-white-intense}{HTML}{A1A6B2}

    % commands and environments needed by pandoc snippets
    % extracted from the output of `pandoc -s`
    \providecommand{\tightlist}{%
      \setlength{\itemsep}{0pt}\setlength{\parskip}{0pt}}
    \DefineVerbatimEnvironment{Highlighting}{Verbatim}{commandchars=\\\{\}}
    % Add ',fontsize=\small' for more characters per line
    \newenvironment{Shaded}{}{}
    \newcommand{\KeywordTok}[1]{\textcolor[rgb]{0.00,0.44,0.13}{\textbf{{#1}}}}
    \newcommand{\DataTypeTok}[1]{\textcolor[rgb]{0.56,0.13,0.00}{{#1}}}
    \newcommand{\DecValTok}[1]{\textcolor[rgb]{0.25,0.63,0.44}{{#1}}}
    \newcommand{\BaseNTok}[1]{\textcolor[rgb]{0.25,0.63,0.44}{{#1}}}
    \newcommand{\FloatTok}[1]{\textcolor[rgb]{0.25,0.63,0.44}{{#1}}}
    \newcommand{\CharTok}[1]{\textcolor[rgb]{0.25,0.44,0.63}{{#1}}}
    \newcommand{\StringTok}[1]{\textcolor[rgb]{0.25,0.44,0.63}{{#1}}}
    \newcommand{\CommentTok}[1]{\textcolor[rgb]{0.38,0.63,0.69}{\textit{{#1}}}}
    \newcommand{\OtherTok}[1]{\textcolor[rgb]{0.00,0.44,0.13}{{#1}}}
    \newcommand{\AlertTok}[1]{\textcolor[rgb]{1.00,0.00,0.00}{\textbf{{#1}}}}
    \newcommand{\FunctionTok}[1]{\textcolor[rgb]{0.02,0.16,0.49}{{#1}}}
    \newcommand{\RegionMarkerTok}[1]{{#1}}
    \newcommand{\ErrorTok}[1]{\textcolor[rgb]{1.00,0.00,0.00}{\textbf{{#1}}}}
    \newcommand{\NormalTok}[1]{{#1}}
    
    % Additional commands for more recent versions of Pandoc
    \newcommand{\ConstantTok}[1]{\textcolor[rgb]{0.53,0.00,0.00}{{#1}}}
    \newcommand{\SpecialCharTok}[1]{\textcolor[rgb]{0.25,0.44,0.63}{{#1}}}
    \newcommand{\VerbatimStringTok}[1]{\textcolor[rgb]{0.25,0.44,0.63}{{#1}}}
    \newcommand{\SpecialStringTok}[1]{\textcolor[rgb]{0.73,0.40,0.53}{{#1}}}
    \newcommand{\ImportTok}[1]{{#1}}
    \newcommand{\DocumentationTok}[1]{\textcolor[rgb]{0.73,0.13,0.13}{\textit{{#1}}}}
    \newcommand{\AnnotationTok}[1]{\textcolor[rgb]{0.38,0.63,0.69}{\textbf{\textit{{#1}}}}}
    \newcommand{\CommentVarTok}[1]{\textcolor[rgb]{0.38,0.63,0.69}{\textbf{\textit{{#1}}}}}
    \newcommand{\VariableTok}[1]{\textcolor[rgb]{0.10,0.09,0.49}{{#1}}}
    \newcommand{\ControlFlowTok}[1]{\textcolor[rgb]{0.00,0.44,0.13}{\textbf{{#1}}}}
    \newcommand{\OperatorTok}[1]{\textcolor[rgb]{0.40,0.40,0.40}{{#1}}}
    \newcommand{\BuiltInTok}[1]{{#1}}
    \newcommand{\ExtensionTok}[1]{{#1}}
    \newcommand{\PreprocessorTok}[1]{\textcolor[rgb]{0.74,0.48,0.00}{{#1}}}
    \newcommand{\AttributeTok}[1]{\textcolor[rgb]{0.49,0.56,0.16}{{#1}}}
    \newcommand{\InformationTok}[1]{\textcolor[rgb]{0.38,0.63,0.69}{\textbf{\textit{{#1}}}}}
    \newcommand{\WarningTok}[1]{\textcolor[rgb]{0.38,0.63,0.69}{\textbf{\textit{{#1}}}}}
    
    
    % Define a nice break command that doesn't care if a line doesn't already
    % exist.
    \def\br{\hspace*{\fill} \\* }
    % Math Jax compatability definitions
    \def\gt{>}
    \def\lt{<}
    % Document parameters
    \title{espacio-ipv4-no-anunciado-v1}
    
    
    

    % Pygments definitions
    
\makeatletter
\def\PY@reset{\let\PY@it=\relax \let\PY@bf=\relax%
    \let\PY@ul=\relax \let\PY@tc=\relax%
    \let\PY@bc=\relax \let\PY@ff=\relax}
\def\PY@tok#1{\csname PY@tok@#1\endcsname}
\def\PY@toks#1+{\ifx\relax#1\empty\else%
    \PY@tok{#1}\expandafter\PY@toks\fi}
\def\PY@do#1{\PY@bc{\PY@tc{\PY@ul{%
    \PY@it{\PY@bf{\PY@ff{#1}}}}}}}
\def\PY#1#2{\PY@reset\PY@toks#1+\relax+\PY@do{#2}}

\expandafter\def\csname PY@tok@w\endcsname{\def\PY@tc##1{\textcolor[rgb]{0.73,0.73,0.73}{##1}}}
\expandafter\def\csname PY@tok@c\endcsname{\let\PY@it=\textit\def\PY@tc##1{\textcolor[rgb]{0.25,0.50,0.50}{##1}}}
\expandafter\def\csname PY@tok@cp\endcsname{\def\PY@tc##1{\textcolor[rgb]{0.74,0.48,0.00}{##1}}}
\expandafter\def\csname PY@tok@k\endcsname{\let\PY@bf=\textbf\def\PY@tc##1{\textcolor[rgb]{0.00,0.50,0.00}{##1}}}
\expandafter\def\csname PY@tok@kp\endcsname{\def\PY@tc##1{\textcolor[rgb]{0.00,0.50,0.00}{##1}}}
\expandafter\def\csname PY@tok@kt\endcsname{\def\PY@tc##1{\textcolor[rgb]{0.69,0.00,0.25}{##1}}}
\expandafter\def\csname PY@tok@o\endcsname{\def\PY@tc##1{\textcolor[rgb]{0.40,0.40,0.40}{##1}}}
\expandafter\def\csname PY@tok@ow\endcsname{\let\PY@bf=\textbf\def\PY@tc##1{\textcolor[rgb]{0.67,0.13,1.00}{##1}}}
\expandafter\def\csname PY@tok@nb\endcsname{\def\PY@tc##1{\textcolor[rgb]{0.00,0.50,0.00}{##1}}}
\expandafter\def\csname PY@tok@nf\endcsname{\def\PY@tc##1{\textcolor[rgb]{0.00,0.00,1.00}{##1}}}
\expandafter\def\csname PY@tok@nc\endcsname{\let\PY@bf=\textbf\def\PY@tc##1{\textcolor[rgb]{0.00,0.00,1.00}{##1}}}
\expandafter\def\csname PY@tok@nn\endcsname{\let\PY@bf=\textbf\def\PY@tc##1{\textcolor[rgb]{0.00,0.00,1.00}{##1}}}
\expandafter\def\csname PY@tok@ne\endcsname{\let\PY@bf=\textbf\def\PY@tc##1{\textcolor[rgb]{0.82,0.25,0.23}{##1}}}
\expandafter\def\csname PY@tok@nv\endcsname{\def\PY@tc##1{\textcolor[rgb]{0.10,0.09,0.49}{##1}}}
\expandafter\def\csname PY@tok@no\endcsname{\def\PY@tc##1{\textcolor[rgb]{0.53,0.00,0.00}{##1}}}
\expandafter\def\csname PY@tok@nl\endcsname{\def\PY@tc##1{\textcolor[rgb]{0.63,0.63,0.00}{##1}}}
\expandafter\def\csname PY@tok@ni\endcsname{\let\PY@bf=\textbf\def\PY@tc##1{\textcolor[rgb]{0.60,0.60,0.60}{##1}}}
\expandafter\def\csname PY@tok@na\endcsname{\def\PY@tc##1{\textcolor[rgb]{0.49,0.56,0.16}{##1}}}
\expandafter\def\csname PY@tok@nt\endcsname{\let\PY@bf=\textbf\def\PY@tc##1{\textcolor[rgb]{0.00,0.50,0.00}{##1}}}
\expandafter\def\csname PY@tok@nd\endcsname{\def\PY@tc##1{\textcolor[rgb]{0.67,0.13,1.00}{##1}}}
\expandafter\def\csname PY@tok@s\endcsname{\def\PY@tc##1{\textcolor[rgb]{0.73,0.13,0.13}{##1}}}
\expandafter\def\csname PY@tok@sd\endcsname{\let\PY@it=\textit\def\PY@tc##1{\textcolor[rgb]{0.73,0.13,0.13}{##1}}}
\expandafter\def\csname PY@tok@si\endcsname{\let\PY@bf=\textbf\def\PY@tc##1{\textcolor[rgb]{0.73,0.40,0.53}{##1}}}
\expandafter\def\csname PY@tok@se\endcsname{\let\PY@bf=\textbf\def\PY@tc##1{\textcolor[rgb]{0.73,0.40,0.13}{##1}}}
\expandafter\def\csname PY@tok@sr\endcsname{\def\PY@tc##1{\textcolor[rgb]{0.73,0.40,0.53}{##1}}}
\expandafter\def\csname PY@tok@ss\endcsname{\def\PY@tc##1{\textcolor[rgb]{0.10,0.09,0.49}{##1}}}
\expandafter\def\csname PY@tok@sx\endcsname{\def\PY@tc##1{\textcolor[rgb]{0.00,0.50,0.00}{##1}}}
\expandafter\def\csname PY@tok@m\endcsname{\def\PY@tc##1{\textcolor[rgb]{0.40,0.40,0.40}{##1}}}
\expandafter\def\csname PY@tok@gh\endcsname{\let\PY@bf=\textbf\def\PY@tc##1{\textcolor[rgb]{0.00,0.00,0.50}{##1}}}
\expandafter\def\csname PY@tok@gu\endcsname{\let\PY@bf=\textbf\def\PY@tc##1{\textcolor[rgb]{0.50,0.00,0.50}{##1}}}
\expandafter\def\csname PY@tok@gd\endcsname{\def\PY@tc##1{\textcolor[rgb]{0.63,0.00,0.00}{##1}}}
\expandafter\def\csname PY@tok@gi\endcsname{\def\PY@tc##1{\textcolor[rgb]{0.00,0.63,0.00}{##1}}}
\expandafter\def\csname PY@tok@gr\endcsname{\def\PY@tc##1{\textcolor[rgb]{1.00,0.00,0.00}{##1}}}
\expandafter\def\csname PY@tok@ge\endcsname{\let\PY@it=\textit}
\expandafter\def\csname PY@tok@gs\endcsname{\let\PY@bf=\textbf}
\expandafter\def\csname PY@tok@gp\endcsname{\let\PY@bf=\textbf\def\PY@tc##1{\textcolor[rgb]{0.00,0.00,0.50}{##1}}}
\expandafter\def\csname PY@tok@go\endcsname{\def\PY@tc##1{\textcolor[rgb]{0.53,0.53,0.53}{##1}}}
\expandafter\def\csname PY@tok@gt\endcsname{\def\PY@tc##1{\textcolor[rgb]{0.00,0.27,0.87}{##1}}}
\expandafter\def\csname PY@tok@err\endcsname{\def\PY@bc##1{\setlength{\fboxsep}{0pt}\fcolorbox[rgb]{1.00,0.00,0.00}{1,1,1}{\strut ##1}}}
\expandafter\def\csname PY@tok@kc\endcsname{\let\PY@bf=\textbf\def\PY@tc##1{\textcolor[rgb]{0.00,0.50,0.00}{##1}}}
\expandafter\def\csname PY@tok@kd\endcsname{\let\PY@bf=\textbf\def\PY@tc##1{\textcolor[rgb]{0.00,0.50,0.00}{##1}}}
\expandafter\def\csname PY@tok@kn\endcsname{\let\PY@bf=\textbf\def\PY@tc##1{\textcolor[rgb]{0.00,0.50,0.00}{##1}}}
\expandafter\def\csname PY@tok@kr\endcsname{\let\PY@bf=\textbf\def\PY@tc##1{\textcolor[rgb]{0.00,0.50,0.00}{##1}}}
\expandafter\def\csname PY@tok@bp\endcsname{\def\PY@tc##1{\textcolor[rgb]{0.00,0.50,0.00}{##1}}}
\expandafter\def\csname PY@tok@fm\endcsname{\def\PY@tc##1{\textcolor[rgb]{0.00,0.00,1.00}{##1}}}
\expandafter\def\csname PY@tok@vc\endcsname{\def\PY@tc##1{\textcolor[rgb]{0.10,0.09,0.49}{##1}}}
\expandafter\def\csname PY@tok@vg\endcsname{\def\PY@tc##1{\textcolor[rgb]{0.10,0.09,0.49}{##1}}}
\expandafter\def\csname PY@tok@vi\endcsname{\def\PY@tc##1{\textcolor[rgb]{0.10,0.09,0.49}{##1}}}
\expandafter\def\csname PY@tok@vm\endcsname{\def\PY@tc##1{\textcolor[rgb]{0.10,0.09,0.49}{##1}}}
\expandafter\def\csname PY@tok@sa\endcsname{\def\PY@tc##1{\textcolor[rgb]{0.73,0.13,0.13}{##1}}}
\expandafter\def\csname PY@tok@sb\endcsname{\def\PY@tc##1{\textcolor[rgb]{0.73,0.13,0.13}{##1}}}
\expandafter\def\csname PY@tok@sc\endcsname{\def\PY@tc##1{\textcolor[rgb]{0.73,0.13,0.13}{##1}}}
\expandafter\def\csname PY@tok@dl\endcsname{\def\PY@tc##1{\textcolor[rgb]{0.73,0.13,0.13}{##1}}}
\expandafter\def\csname PY@tok@s2\endcsname{\def\PY@tc##1{\textcolor[rgb]{0.73,0.13,0.13}{##1}}}
\expandafter\def\csname PY@tok@sh\endcsname{\def\PY@tc##1{\textcolor[rgb]{0.73,0.13,0.13}{##1}}}
\expandafter\def\csname PY@tok@s1\endcsname{\def\PY@tc##1{\textcolor[rgb]{0.73,0.13,0.13}{##1}}}
\expandafter\def\csname PY@tok@mb\endcsname{\def\PY@tc##1{\textcolor[rgb]{0.40,0.40,0.40}{##1}}}
\expandafter\def\csname PY@tok@mf\endcsname{\def\PY@tc##1{\textcolor[rgb]{0.40,0.40,0.40}{##1}}}
\expandafter\def\csname PY@tok@mh\endcsname{\def\PY@tc##1{\textcolor[rgb]{0.40,0.40,0.40}{##1}}}
\expandafter\def\csname PY@tok@mi\endcsname{\def\PY@tc##1{\textcolor[rgb]{0.40,0.40,0.40}{##1}}}
\expandafter\def\csname PY@tok@il\endcsname{\def\PY@tc##1{\textcolor[rgb]{0.40,0.40,0.40}{##1}}}
\expandafter\def\csname PY@tok@mo\endcsname{\def\PY@tc##1{\textcolor[rgb]{0.40,0.40,0.40}{##1}}}
\expandafter\def\csname PY@tok@ch\endcsname{\let\PY@it=\textit\def\PY@tc##1{\textcolor[rgb]{0.25,0.50,0.50}{##1}}}
\expandafter\def\csname PY@tok@cm\endcsname{\let\PY@it=\textit\def\PY@tc##1{\textcolor[rgb]{0.25,0.50,0.50}{##1}}}
\expandafter\def\csname PY@tok@cpf\endcsname{\let\PY@it=\textit\def\PY@tc##1{\textcolor[rgb]{0.25,0.50,0.50}{##1}}}
\expandafter\def\csname PY@tok@c1\endcsname{\let\PY@it=\textit\def\PY@tc##1{\textcolor[rgb]{0.25,0.50,0.50}{##1}}}
\expandafter\def\csname PY@tok@cs\endcsname{\let\PY@it=\textit\def\PY@tc##1{\textcolor[rgb]{0.25,0.50,0.50}{##1}}}

\def\PYZbs{\char`\\}
\def\PYZus{\char`\_}
\def\PYZob{\char`\{}
\def\PYZcb{\char`\}}
\def\PYZca{\char`\^}
\def\PYZam{\char`\&}
\def\PYZlt{\char`\<}
\def\PYZgt{\char`\>}
\def\PYZsh{\char`\#}
\def\PYZpc{\char`\%}
\def\PYZdl{\char`\$}
\def\PYZhy{\char`\-}
\def\PYZsq{\char`\'}
\def\PYZdq{\char`\"}
\def\PYZti{\char`\~}
% for compatibility with earlier versions
\def\PYZat{@}
\def\PYZlb{[}
\def\PYZrb{]}
\makeatother


    % Exact colors from NB
    \definecolor{incolor}{rgb}{0.0, 0.0, 0.5}
    \definecolor{outcolor}{rgb}{0.545, 0.0, 0.0}



    
    % Prevent overflowing lines due to hard-to-break entities
    \sloppy 
    % Setup hyperref package
    \hypersetup{
      breaklinks=true,  % so long urls are correctly broken across lines
      colorlinks=true,
      urlcolor=urlcolor,
      linkcolor=linkcolor,
      citecolor=citecolor,
      }
    % Slightly bigger margins than the latex defaults
    
    \geometry{verbose,tmargin=1in,bmargin=1in,lmargin=1in,rmargin=1in}
    
    

    \begin{document}
    
    
    \maketitle
    
    

    
    \hypertarget{reporte-de-espacio-ipv4-no-anunciado}{%
\section{Reporte de espacio IPv4 no
anunciado}\label{reporte-de-espacio-ipv4-no-anunciado}}

Este reporte incluye información acerca de espacio IPv4 asignad por
LACNIC pero no visible en la tabla de enrutamiento global.

\hypertarget{metodologuxeda}{%
\subsection{Metodología}\label{metodologuxeda}}

Se toman las siguientes fuentes de datos:

\begin{itemize}
\tightlist
\item
  ``delegated-extended'' de LACNIC
\item
  RIPE RIS IPv4 Dump
\end{itemize}

El algoritmo de alto nivel es:

\begin{enumerate}
\def\labelenumi{\arabic{enumi}.}
\tightlist
\item
  Recorrer el delegated extended

  \begin{enumerate}
  \def\labelenumii{\arabic{enumii}.}
  \tightlist
  \item
    Para cada organización, tomar todas sus asignaciones
  \item
    Producir un agregado de las mismas
  \end{enumerate}
\item
  Recorrer el producto de (1)

  \begin{enumerate}
  \def\labelenumii{\arabic{enumii}.}
  \tightlist
  \item
    Para cada ruta de RIS, identificar el agregado de asignaciones de
    donde proviene, y de allí la organización a la que pertenece
  \end{enumerate}
\item
  Recorrer el producto de (2)

  \begin{enumerate}
  \def\labelenumii{\arabic{enumii}.}
  \tightlist
  \item
    Para cada agregado, agregar todas las rutas que le corresponden y de
    allí calcular cuanto del total todas esas rutas cubren.
  \end{enumerate}
\end{enumerate}

\hypertarget{producto-del-algoritmo}{%
\subsection{Producto del algoritmo}\label{producto-del-algoritmo}}

El producto es un archivo separado por pipes con la siguiente
estructura:

\begin{verbatim}
orgid|prefix|visible|dark|total
20118|24.232.0.0/16|65536|0|65536
272078|45.4.0.0/22|1024|0|1024
271038|45.4.4.0/22|1024|0|1024
261330|45.4.8.0/22|1024|0|1024

...
...
258668|45.4.60.0/22|1024|0|1024
258825|45.4.64.0/22|1024|0|1024
\end{verbatim}

Las columnas:

\begin{itemize}
\tightlist
\item
  orgid: coincide con el opaque-id del delegated-extended, identifica la
  organización
\item
  prefijo: prefijo IPv4
\item
  visible: cantidad de direcciones de este prefijo visibles (es decir,
  contenidas en rutas)
\item
  dark: cantidad de direcciones no-visibles
\item
  total: cantidad total de direcciones contenidas en el prefijo
\end{itemize}

    \hypertarget{resultados}{%
\subsection{Resultados}\label{resultados}}

\hypertarget{r1-cantidad-de-asignaciones-completamente-invisibles-por-organizaciuxf3n}{%
\subsubsection{R1: Cantidad de asignaciones completamente invisibles,
por
organización}\label{r1-cantidad-de-asignaciones-completamente-invisibles-por-organizaciuxf3n}}

En este resultado se listan las asignaciones completamente invisibles
(es decir, sin ninguna ruta que las cubra) y las organizaciones que las
tienen.

    \begin{Verbatim}[commandchars=\\\{\}]
{\color{incolor}In [{\color{incolor}56}]:} \PY{k+kn}{import} \PY{n+nn}{pandas}
         \PY{k+kn}{import} \PY{n+nn}{sqlite3}
         
         \PY{n}{w\PYZus{}date} \PY{o}{=} \PY{l+s+s1}{\PYZsq{}}\PY{l+s+s1}{20191110}\PY{l+s+s1}{\PYZsq{}}
         \PY{n}{conn} \PY{o}{=} \PY{n}{sqlite3}\PY{o}{.}\PY{n}{connect}\PY{p}{(}\PY{l+s+s2}{\PYZdq{}}\PY{l+s+s2}{:memory:}\PY{l+s+s2}{\PYZdq{}}\PY{p}{)}
         
         \PY{n}{inf} \PY{o}{=} \PY{l+s+s2}{\PYZdq{}}\PY{l+s+s2}{s1\PYZus{}invisible\PYZus{}prefixes\PYZhy{}}\PY{l+s+si}{\PYZob{}\PYZcb{}}\PY{l+s+s2}{.csv}\PY{l+s+s2}{\PYZdq{}}\PY{o}{.}\PY{n}{format}\PY{p}{(}\PY{n}{w\PYZus{}date}\PY{p}{)}
         \PY{n}{df} \PY{o}{=} \PY{n}{pandas}\PY{o}{.}\PY{n}{read\PYZus{}csv}\PY{p}{(}\PY{n}{inf}\PY{p}{,} \PY{n}{delimiter}\PY{o}{=}\PY{l+s+s1}{\PYZsq{}}\PY{l+s+s1}{|}\PY{l+s+s1}{\PYZsq{}}\PY{p}{)}
         \PY{n}{orgs} \PY{o}{=} \PY{n}{pandas}\PY{o}{.}\PY{n}{read\PYZus{}csv}\PY{p}{(}\PY{l+s+s2}{\PYZdq{}}\PY{l+s+s2}{orgs\PYZus{}lacnic.csv}\PY{l+s+s2}{\PYZdq{}}\PY{p}{)}
         \PY{n}{df}\PY{o}{.}\PY{n}{to\PYZus{}sql}\PY{p}{(}\PY{l+s+s2}{\PYZdq{}}\PY{l+s+s2}{invr}\PY{l+s+s2}{\PYZdq{}}\PY{p}{,} \PY{n}{conn}\PY{p}{,} \PY{n}{if\PYZus{}exists}\PY{o}{=}\PY{l+s+s1}{\PYZsq{}}\PY{l+s+s1}{append}\PY{l+s+s1}{\PYZsq{}}\PY{p}{,} \PY{n}{index}\PY{o}{=}\PY{k+kc}{False}\PY{p}{)}
         
         \PY{n}{df}\PY{o}{.}\PY{n}{head}\PY{p}{(}\PY{p}{)}
\end{Verbatim}


\begin{Verbatim}[commandchars=\\\{\}]
{\color{outcolor}Out[{\color{outcolor}56}]:}     orgid         prefix  visible  dark  total
         0   20118  24.232.0.0/16    65536     0  65536
         1  272078    45.4.0.0/22     1024     0   1024
         2  271038    45.4.4.0/22     1024     0   1024
         3  261330    45.4.8.0/22     1024     0   1024
         4  265673   45.4.12.0/22     1024     0   1024
\end{Verbatim}
            
    \begin{Verbatim}[commandchars=\\\{\}]
{\color{incolor}In [{\color{incolor}57}]:} \PY{n}{orgs}\PY{o}{.}\PY{n}{query}\PY{p}{(}\PY{l+s+s2}{\PYZdq{}}\PY{l+s+s2}{orgid == 65334}\PY{l+s+s2}{\PYZdq{}}\PY{p}{)}
\end{Verbatim}


\begin{Verbatim}[commandchars=\\\{\}]
{\color{outcolor}Out[{\color{outcolor}57}]:}        orgid                                  name           handle country  \textbackslash{}
         45293  65334  United Nations Development Programme  AR-UNDP1-LACNIC      AR   
         
                        city  
         45293  Buenos Aires  
\end{Verbatim}
            
    \hypertarget{r2.-cantidad-de-asignaciones-invisibles-mayores-o-iguales-a-un-16}{%
\subsubsection{R2. Cantidad de asignaciones invisibles mayores o iguales
a un
/16}\label{r2.-cantidad-de-asignaciones-invisibles-mayores-o-iguales-a-un-16}}

    \begin{Verbatim}[commandchars=\\\{\}]
{\color{incolor}In [{\color{incolor}58}]:} \PY{n}{nva} \PY{o}{=} \PY{n}{df}\PY{o}{.}\PY{n}{query}\PY{p}{(}\PY{l+s+s2}{\PYZdq{}}\PY{l+s+s2}{visible == 0 and total \PYZgt{} 16384}\PY{l+s+s2}{\PYZdq{}}\PY{p}{)}\PY{o}{.}\PY{n}{sort\PYZus{}values}\PY{p}{(}\PY{l+s+s2}{\PYZdq{}}\PY{l+s+s2}{total}\PY{l+s+s2}{\PYZdq{}}\PY{p}{,} \PY{n}{ascending}\PY{o}{=}\PY{k+kc}{False}\PY{p}{)}
         \PY{c+c1}{\PYZsh{} nva.head(20)}
         \PY{n+nb}{print}\PY{p}{(}\PY{l+s+s2}{\PYZdq{}}\PY{l+s+s2}{Cantidad de asignaciones grandes no visibles: }\PY{l+s+si}{\PYZob{}\PYZcb{}}\PY{l+s+s2}{\PYZdq{}}\PY{o}{.}\PY{n}{format}\PY{p}{(}\PY{n+nb}{len}\PY{p}{(}\PY{n}{nva}\PY{p}{[}\PY{l+s+s1}{\PYZsq{}}\PY{l+s+s1}{total}\PY{l+s+s1}{\PYZsq{}}\PY{p}{]}\PY{p}{)}\PY{p}{)}\PY{p}{)}
\end{Verbatim}


    \begin{Verbatim}[commandchars=\\\{\}]
Cantidad de asignaciones grandes no visibles: 26

    \end{Verbatim}

    \begin{Verbatim}[commandchars=\\\{\}]
{\color{incolor}In [{\color{incolor}59}]:} \PY{n+nb}{print}\PY{p}{(}\PY{l+s+s2}{\PYZdq{}}\PY{l+s+s2}{Valor aproximado de mercado de las direcciones no visibles en bloques grandes: }\PY{l+s+si}{\PYZob{}\PYZcb{}}\PY{l+s+s2}{ usd}\PY{l+s+s2}{\PYZdq{}}\PY{o}{.}\PY{n}{format}\PY{p}{(}\PY{n}{nva}\PY{p}{[}\PY{l+s+s1}{\PYZsq{}}\PY{l+s+s1}{total}\PY{l+s+s1}{\PYZsq{}}\PY{p}{]}\PY{o}{.}\PY{n}{sum}\PY{p}{(}\PY{n}{axis}\PY{o}{=}\PY{l+m+mi}{0}\PY{p}{)}\PY{o}{*}\PY{l+m+mi}{15}\PY{p}{)}\PY{p}{)}
\end{Verbatim}


    \begin{Verbatim}[commandchars=\\\{\}]
Valor aproximado de mercado de las direcciones no visibles en bloques grandes: 28508160 usd

    \end{Verbatim}

    \hypertarget{r2.1-listado-de-asignaciones-grandes-no-visibles}{%
\subsubsection{R2.1: Listado de asignaciones grandes no
visibles}\label{r2.1-listado-de-asignaciones-grandes-no-visibles}}

    \begin{Verbatim}[commandchars=\\\{\}]
{\color{incolor}In [{\color{incolor}60}]:} \PY{n}{nva}\PY{o}{.}\PY{n}{to\PYZus{}csv}\PY{p}{(}\PY{l+s+s2}{\PYZdq{}}\PY{l+s+s2}{s1\PYZus{}bigallocs\PYZus{}notvisible\PYZhy{}}\PY{l+s+si}{\PYZob{}\PYZcb{}}\PY{l+s+s2}{.csv}\PY{l+s+s2}{\PYZdq{}}\PY{o}{.}\PY{n}{format}\PY{p}{(}\PY{n}{w\PYZus{}date}\PY{p}{,} \PY{n}{delimiter}\PY{o}{=}\PY{l+s+s1}{\PYZsq{}}\PY{l+s+s1}{|}\PY{l+s+s1}{\PYZsq{}}\PY{p}{)}\PY{p}{)}
         
         \PY{c+c1}{\PYZsh{} orgs[\PYZsq{}orgid\PYZsq{}] = pandas.to\PYZus{}numeric(orgs[\PYZsq{}orgid\PYZsq{}])}
         
         \PY{n}{nva\PYZus{}b} \PY{o}{=} \PY{n}{nva}\PY{o}{.}\PY{n}{merge}\PY{p}{(}\PY{n}{orgs}\PY{p}{,} \PY{n}{on}\PY{o}{=}\PY{l+s+s1}{\PYZsq{}}\PY{l+s+s1}{orgid}\PY{l+s+s1}{\PYZsq{}}\PY{p}{,} \PY{n}{how}\PY{o}{=}\PY{l+s+s1}{\PYZsq{}}\PY{l+s+s1}{left}\PY{l+s+s1}{\PYZsq{}}\PY{p}{)}
         
         \PY{n}{nva\PYZus{}b}\PY{p}{[}\PY{p}{[}\PY{l+s+s1}{\PYZsq{}}\PY{l+s+s1}{prefix}\PY{l+s+s1}{\PYZsq{}}\PY{p}{,} \PY{l+s+s1}{\PYZsq{}}\PY{l+s+s1}{dark}\PY{l+s+s1}{\PYZsq{}}\PY{p}{,}\PY{l+s+s1}{\PYZsq{}}\PY{l+s+s1}{visible}\PY{l+s+s1}{\PYZsq{}}\PY{p}{,} \PY{l+s+s1}{\PYZsq{}}\PY{l+s+s1}{handle}\PY{l+s+s1}{\PYZsq{}}\PY{p}{,}\PY{l+s+s1}{\PYZsq{}}\PY{l+s+s1}{name}\PY{l+s+s1}{\PYZsq{}}\PY{p}{,}\PY{l+s+s1}{\PYZsq{}}\PY{l+s+s1}{country}\PY{l+s+s1}{\PYZsq{}}\PY{p}{]}\PY{p}{]}
\end{Verbatim}


\begin{Verbatim}[commandchars=\\\{\}]
{\color{outcolor}Out[{\color{outcolor}60}]:}               prefix    dark  visible            handle  \textbackslash{}
         0     186.172.0.0/14  262144        0  CL-CCSA39-LACNIC   
         1     186.164.0.0/15  131072        0    VE-TCCA-LACNIC   
         2      186.98.0.0/15  131072        0    CO-CTSE-LACNIC   
         3      129.90.0.0/16   65536        0   VE-INSA1-LACNIC   
         4     163.250.0.0/16   65536        0   CL-RESA2-LACNIC   
         5     186.157.0.0/16   65536        0   AR-CCTI1-LACNIC   
         6     167.134.0.0/16   65536        0    VE-PESA-LACNIC   
         7      167.28.0.0/16   65536        0   CL-GLSA2-LACNIC   
         8     166.110.0.0/16   65536        0    CL-BAED-LACNIC   
         9     140.191.0.0/16   65536        0   AR-UNDP1-LACNIC   
         10     164.96.0.0/16   65536        0    CL-SRCI-LACNIC   
         11     164.98.0.0/16   65536        0    CL-EJCH-LACNIC   
         12    162.122.0.0/16   65536        0    VE-MASA-LACNIC   
         13    161.255.0.0/16   65536        0    VE-TCCA-LACNIC   
         14    161.234.0.0/16   65536        0    VE-TCCA-LACNIC   
         15    158.160.0.0/16   65536        0   VE-MASA1-LACNIC   
         16    155.211.0.0/16   65536        0   BR-CTAE1-LACNIC   
         17    152.139.0.0/16   65536        0   CL-BACH2-LACNIC   
         18    148.250.0.0/16   65536        0    MX-SAWM-LACNIC   
         19    148.248.0.0/16   65536        0   MX-GFIS4-LACNIC   
         20    148.242.0.0/16   65536        0    MX-BNMS-LACNIC   
         21    201.169.0.0/16   65536        0   MX-ASCV9-LACNIC   
         22    177.234.0.0/17   32768        0  MX-SCYT11-LACNIC   
         23    186.103.0.0/17   32768        0    CO-CTSE-LACNIC   
         24    189.201.0.0/17   32768        0    MX-TPTE-LACNIC   
         25  190.197.128.0/17   32768        0    AR-MOBE-LACNIC   
         
                                                          name country  
         0                                    CLARO CHILE S.A.      CL  
         1                         TELEFONICA VENEZOLANA, C.A.      VE  
         2                COLOMBIA TELECOMUNICACIONES S.A. ESP      CO  
         3                                        INTEVEP S.A.      VE  
         4                                        REDBANC S.A.      CL  
         5                                  AMX Argentina S.A.      AR  
         6                                       Pequiven S.A.      VE  
         7                           BANCO DEL ESTADO DE CHILE      CL  
         8                                 Banco de A. Edwards      CL  
         9                United Nations Development Programme      AR  
         10        Servicio de Registro Civil e Identificacion      CL  
         11                                  Ejercito de Chile      CL  
         12                                       Maraven S.A.      VE  
         13                        TELEFONICA VENEZOLANA, C.A.      VE  
         14                        TELEFONICA VENEZOLANA, C.A.      VE  
         15                                        Mavesa s.a.      VE  
         16                        CENTRO TECNICO AEROESPACIAL      BR  
         17                                     Banco de Chile      CL  
         18    SERVICIOS ADMINISTRATIVOS WALMART S DE RL DE CV      MX  
         19          Grupo Financiero Invermexico S.A. de C.V.      MX  
         20                     Banco Nacional de Mexico, S.A.      MX  
         21                              Axtel, S.A.B. de C.V.      MX  
         22  SECRETARIA DE COMUNICACIONES Y TRANSPORTES COO{\ldots}      MX  
         23               COLOMBIA TELECOMUNICACIONES S.A. ESP      CO  
         24             TOTAL PLAY TELECOMUNICACIONES SA DE CV      MX  
         25  Telefónica Móviles Argentina S.A. (Movistar Ar{\ldots}      AR  
\end{Verbatim}
            
    \hypertarget{r3.-porcentaje-de-espacio-no-visible-respecto-del-espacio-asignado}{%
\subsubsection{R3. Porcentaje de espacio no visible respecto del espacio
asignado}\label{r3.-porcentaje-de-espacio-no-visible-respecto-del-espacio-asignado}}

    \begin{Verbatim}[commandchars=\\\{\}]
{\color{incolor}In [{\color{incolor}61}]:} \PY{n}{dark\PYZus{}t} \PY{o}{=} \PY{n}{df}\PY{p}{[}\PY{l+s+s1}{\PYZsq{}}\PY{l+s+s1}{dark}\PY{l+s+s1}{\PYZsq{}}\PY{p}{]}\PY{o}{.}\PY{n}{sum}\PY{p}{(}\PY{p}{)}
         \PY{n}{visible\PYZus{}t} \PY{o}{=} \PY{n}{df}\PY{p}{[}\PY{l+s+s1}{\PYZsq{}}\PY{l+s+s1}{visible}\PY{l+s+s1}{\PYZsq{}}\PY{p}{]}\PY{o}{.}\PY{n}{sum}\PY{p}{(}\PY{p}{)}
         \PY{n}{total\PYZus{}t} \PY{o}{=} \PY{n}{df}\PY{p}{[}\PY{l+s+s1}{\PYZsq{}}\PY{l+s+s1}{total}\PY{l+s+s1}{\PYZsq{}}\PY{p}{]}\PY{o}{.}\PY{n}{sum}\PY{p}{(}\PY{p}{)}
         
         \PY{n+nb}{print}\PY{p}{(}\PY{l+s+s2}{\PYZdq{}}\PY{l+s+s2}{El porcentaje de espacio asignado pero no visible: }\PY{l+s+si}{\PYZob{}0:.2f\PYZcb{}}\PY{l+s+s2}{\PYZpc{}}\PY{l+s+s2}{\PYZdq{}}\PY{o}{.}\PY{n}{format}\PY{p}{(}\PY{n}{dark\PYZus{}t} \PY{o}{/} \PY{n}{total\PYZus{}t} \PY{o}{*} \PY{l+m+mi}{100}\PY{p}{)}\PY{p}{)}
\end{Verbatim}


    \begin{Verbatim}[commandchars=\\\{\}]
El porcentaje de espacio asignado pero no visible: 8.06\%

    \end{Verbatim}

    \hypertarget{r4.-cantidad-de-asignaciones-con-mayor-espacio-oscuro-pero-parcialmente-visibles}{%
\subsubsection{R4. Cantidad de asignaciones con mayor espacio oscuro,
pero parcialmente
visibles}\label{r4.-cantidad-de-asignaciones-con-mayor-espacio-oscuro-pero-parcialmente-visibles}}

    \begin{Verbatim}[commandchars=\\\{\}]
{\color{incolor}In [{\color{incolor}62}]:} \PY{n}{nva\PYZus{}parcial} \PY{o}{=} \PY{n}{df}\PY{o}{.}\PY{n}{query}\PY{p}{(}\PY{l+s+s2}{\PYZdq{}}\PY{l+s+s2}{dark\PYZgt{}0}\PY{l+s+s2}{\PYZdq{}}\PY{p}{)}\PY{o}{.}\PY{n}{sort\PYZus{}values}\PY{p}{(}\PY{l+s+s1}{\PYZsq{}}\PY{l+s+s1}{dark}\PY{l+s+s1}{\PYZsq{}}\PY{p}{,} \PY{n}{ascending}\PY{o}{=}\PY{k+kc}{False}\PY{p}{)}
         \PY{n}{nvb\PYZus{}parcial} \PY{o}{=} \PY{n}{nva\PYZus{}parcial}\PY{o}{.}\PY{n}{merge}\PY{p}{(}\PY{n}{orgs}\PY{p}{,} \PY{n}{on}\PY{o}{=}\PY{l+s+s1}{\PYZsq{}}\PY{l+s+s1}{orgid}\PY{l+s+s1}{\PYZsq{}}\PY{p}{,} \PY{n}{how}\PY{o}{=}\PY{l+s+s1}{\PYZsq{}}\PY{l+s+s1}{left}\PY{l+s+s1}{\PYZsq{}}\PY{p}{)}
         
         \PY{c+c1}{\PYZsh{} }
         
         \PY{n}{nva\PYZus{}parcial\PYZus{}agg} \PY{o}{=} \PY{n}{nva\PYZus{}parcial}\PY{o}{.}\PY{n}{groupby}\PY{p}{(}\PY{l+s+s1}{\PYZsq{}}\PY{l+s+s1}{orgid}\PY{l+s+s1}{\PYZsq{}}\PY{p}{)}
         \PY{n}{nvb\PYZus{}parcial}\PY{o}{.}\PY{n}{sort\PYZus{}values}\PY{p}{(}\PY{l+s+s1}{\PYZsq{}}\PY{l+s+s1}{dark}\PY{l+s+s1}{\PYZsq{}}\PY{p}{,} \PY{n}{ascending}\PY{o}{=}\PY{k+kc}{False}\PY{p}{)}\PY{o}{.}\PY{n}{head}\PY{p}{(}\PY{l+m+mi}{10}\PY{p}{)}
\end{Verbatim}


\begin{Verbatim}[commandchars=\\\{\}]
{\color{outcolor}Out[{\color{outcolor}62}]:}     orgid          prefix  visible    dark    total  \textbackslash{}
         0   21461   201.96.0.0/11  1464064  633088  2097152   
         1   21461  189.128.0.0/10  3690496  503808  4194304   
         2   21461  187.192.0.0/11  1617408  479744  2097152   
         3   21461  187.128.0.0/11  1703936  393216  2097152   
         4   35073  191.104.0.0/13   131072  393216   524288   
         5   21461  187.224.0.0/12   673792  374784  1048576   
         6   20118   181.96.0.0/12   716800  331776  1048576   
         7   21461  189.224.0.0/11  1819136  278016  2097152   
         8  236102  181.216.0.0/13   262144  262144   524288   
         9   47779  186.172.0.0/14        0  262144   262144   
         
                                            name            handle country  \textbackslash{}
         0                   Uninet S.A. de C.V.   MX-USCV4-LACNIC      MX   
         1                   Uninet S.A. de C.V.   MX-USCV4-LACNIC      MX   
         2                   Uninet S.A. de C.V.   MX-USCV4-LACNIC      MX   
         3                   Uninet S.A. de C.V.   MX-USCV4-LACNIC      MX   
         4  COLOMBIA TELECOMUNICACIONES S.A. ESP    CO-CTSE-LACNIC      CO   
         5                   Uninet S.A. de C.V.   MX-USCV4-LACNIC      MX   
         6                Telecom Argentina S.A.    AR-TAST-LACNIC      AR   
         7                   Uninet S.A. de C.V.   MX-USCV4-LACNIC      MX   
         8                            CLARO S.A.   BR-CLSA7-LACNIC      BR   
         9                      CLARO CHILE S.A.  CL-CCSA39-LACNIC      CL   
         
                    city  
         0       Tlalpan  
         1       Tlalpan  
         2       Tlalpan  
         3       Tlalpan  
         4        BOGOTA  
         5       Tlalpan  
         6  Buenos Aires  
         7       Tlalpan  
         8     São Paulo  
         9      Santiago  
\end{Verbatim}
            
    \begin{Verbatim}[commandchars=\\\{\}]
{\color{incolor}In [{\color{incolor}63}]:} \PY{n}{nvb\PYZus{}parcial}\PY{o}{.}\PY{n}{groupby}\PY{p}{(}\PY{l+s+s1}{\PYZsq{}}\PY{l+s+s1}{handle}\PY{l+s+s1}{\PYZsq{}}\PY{p}{)}\PY{o}{.}\PY{n}{sum}\PY{p}{(}\PY{p}{)}\PY{o}{.}\PY{n}{sort\PYZus{}values}\PY{p}{(}\PY{l+s+s1}{\PYZsq{}}\PY{l+s+s1}{dark}\PY{l+s+s1}{\PYZsq{}}\PY{p}{,} \PY{n}{ascending}\PY{o}{=}\PY{k+kc}{False}\PY{p}{)}\PY{o}{.}\PY{n}{head}\PY{p}{(}\PY{l+m+mi}{10}\PY{p}{)}
         \PY{c+c1}{\PYZsh{} nvb\PYZus{}parcial.groupby(\PYZsq{}name\PYZsq{}).head(10)}
\end{Verbatim}


\begin{Verbatim}[commandchars=\\\{\}]
{\color{outcolor}Out[{\color{outcolor}63}]:}                     orgid   visible     dark     total
         handle                                                
         MX-USCV4-LACNIC    493603  12292864  3056384  15349248
         VE-TCCA-LACNIC     300105    546816  1312768   1859584
         CO-CTSE-LACNIC     561168   1666048  1119232   2785280
         AR-TAST-LACNIC     281652   4053504   943616   4997120
         BR-CLSA7-LACNIC   5430346   5158656   706816   5865472
         MX-ASCV9-LACNIC    536375   1162752   492032   1654784
         CL-CCSA39-LACNIC   191116    232448   431104    663552
         EC-ANSA-LACNIC     317450    285184   416256    701440
         MX-CTSC6-LACNIC    691054    643328   302848    946176
         MX-MSCV17-LACNIC   218820   2106112   257280   2363392
\end{Verbatim}
            
    \begin{Verbatim}[commandchars=\\\{\}]
{\color{incolor}In [{\color{incolor}64}]:} \PY{n}{nva\PYZus{}parcial\PYZus{}agg}\PY{o}{.}\PY{n}{get\PYZus{}group}\PY{p}{(}\PY{l+m+mi}{35073}\PY{p}{)}\PY{o}{.}\PY{n}{sum}\PY{p}{(}\PY{p}{)}
\end{Verbatim}


\begin{Verbatim}[commandchars=\\\{\}]
{\color{outcolor}Out[{\color{outcolor}64}]:} orgid                                                 561168
         prefix     191.104.0.0/13186.98.0.0/15186.168.0.0/14152.2{\ldots}
         visible                                              1666048
         dark                                                 1119232
         total                                                2785280
         dtype: object
\end{Verbatim}
            

    % Add a bibliography block to the postdoc
    
    
    
    \end{document}
